\subsection{Experimental Setup}\label{sec:exp-setup}

\textbf{Experiment Cluster}: The experiments were conducted with Hadoop 0.20~\cite{hadoop} on a cluster with 5 machines as worker nodes. Each machine has 8 cores with hyperthreading enabled, 32 GB RAM and 2 TB storage. We set each machine to run 15 concurrent mappers and 10 concurrent reducers.

\textbf{Experiment Data Sets}: There exist several benchmark data sets to assess the accuracy of numerical algorithms and software, such as NIST StRD~\cite{nist}. However, these benchmarks mostly provide very small date sets. For example, the largest data set in NIST StRD contains only 5000 data points. To test the numerical stability of distributed algorithms, we generated large synthetic data sets similar to those in NIST StRD. Our data generator takes the data size and the value range as inputs, and generates values from uniform distribution. For our experiments, we created data sets with different sizes (10million to 1billion) whose values are in the following 3 ranges, R1:[1.0 -- 1.5), R2:[1000.0 -- 1000.5) and R3:[1000000.0 -- 1000000.5).

\textbf{Accuracy Measurement}: In order to assess the numerical accuracy of the results produced by any algorithm, we need the true values of statistics. For this purpose, we rely on Java $BigDecimal$ that can represent arbitrary-precision signed decimal numbers. We implemented the naive algorithms for all statistics using Java $BigDecimal$ with precision $1000$. With such a high precision, results of all mathematically equivalent algorithms should approach closely to the true value. We implemented naive recursive summation for sum, sum divided by count for mean, one-pass algorithms for higher-order statistics as shown in Table~\ref{tab:1-pass}, and textbook one-pass algorithm for covariance. We consider obtained results as the ``true values" of these statistics.

We measure the accuracy achieved by different algorithms using the Log Relative Error (LRE) metric described in~\cite{LRE}. If $q$ is the computed value from an algorithm and $t$ is the true value, then LRE is defined as $$LRE=-\log_{10}|\frac{q-t}{t}|$$
LRE measures the number of significant digits that match between the computed value from the algorithm $q$ and the true value $t$. Therefore, a higher value of LRE indicates that the algorithm is numerically more stable.

%The higher a LRE value is, the better the algorithm. 
%is, the more desirable the estimated value is. As mentioned above, $t$ in our context refers to the values obtained using $BigDecimal$.

\begin{table}[tbh]
\caption{Textbook One-Pass Algorithms for Higher-Order Statistics}
\label{tab:1-pass}
\centering
\begin{tabular}{|c|c|}
\hline
& Equations for 1-Pass Algorithm \\
\hline
variance & $\frac{1}{n-1}S_2-\frac{1}{n(n-1)}S_1^2$ \\
\hline
std & $(variance)^{\frac{1}{2}}$ \\
\hline
skewness & $\frac{S_3 - \frac{3}{n}S_1S_2 + \frac{2}{n^2}S_1^3}{n\times std\times variance}$\\
\hline
kurtosis & $\frac{S_4 - \frac{4}{n}S_3S_1 + \frac{6}{n^2}S_2S_1^2 - \frac{3}{n^3}S_1^4}{n\times (variance)^2} - 3$ \\
\hline
%\multicolumn{2}{l}{}\\
\multicolumn{2}{c}{$S_p=\sum\limits_{i=1}^{n}x_i^p$, which can be easily computed in one pass.}
\end{tabular}
\end{table}

\subsection{Numerical Stability of Univariate Statistics}\label{sec:univariate_exp}

In this section, we demonstrate the numeric stability of SystemML in computing a subset of univariate statistics. 

\begin{table*}[thb]
\caption{Numerical Accuracy of Sum and Mean (LRE values)}
\label{tab:sum}
\centering
\begin{tabular}{|c|c|r|r|r|r|r|}
\hline
 & \textbf{Size} & \multicolumn{3}{|c|}{\textbf{Sum}} &  \multicolumn{2}{|c|}{\textbf{Mean}}\\
 & (million) & \multicolumn{3}{|c|}{} &  \multicolumn{2}{|c|}{}\\
\hline
\textbf{Range} & & SystemML & Naive & Sorted & SystemML & Naive\\ 
\hline
             & 10   & 16.1 & 13.5 & 16.1 & 16.7 & 13.5 \\
\textbf{R1}  & 100  & 16.3 & 13.8 & 13.6 & 16.2 & 13.8 \\
             & 1000 & 16.8 & 13.6 & 13.5 & 16.5 & 13.6 \\
\hline
\hline
             & 10   & 16.8 & 14.4 & 13.9 & 16.5 & 14.4 \\
\textbf{R2}  & 100  & 16.1 & 13.4 & 13.4 & 16.9 & 13.4 \\
             & 1000 & 16.6 & 13.1 & 13.9 & 16.4 & 13.1 \\
\hline
\hline
             & 10   & 15.9 & 14.0 & 13.9 & 16.3 & 14.0 \\
\textbf{R3}  & 100  & 16.0 & 13.1 & 13.4 & 16.9 & 13.1 \\
             & 1000 & 16.3 & 12.9 & 12.2 & 16.5 & 12.9 \\
\hline
\end{tabular}
\end{table*}

Table~\ref{tab:sum} lists the accuracy (LRE values) of results produced by different algorithms for sum and $mean$. For summation, we compare the MapReduce Kahan summation used in SystemML against the naive recursive summation and the sorted summation. The latter two algorithms are adapted to the MapReduce environment. In case of naive recursive summation, mappers compute the partial sums using recursive summation, which are then aggregated in the reducer. 
%This adapted naive recursive summation is used in both PIG and HIVE. 
In case of sorted summation, we first sort the entire data on MapReduce using the algorithm described in Section~\ref{sec:sort-order-stats}, and then apply the adapted naive recursive summation on the sorted data. As shown in Table~\ref{tab:sum}, SystemML consistently produces more accurate results than the other two methods. The accuracies from naive recursive summation and the sorted summation are comparable. In terms of runtime performance, our MapReduce Kahan summation and the naive recursive summation are similar, but the sorted summation is up to $5$ times slower as it performs an explicit sort on the entire data. Similarly, the accuracies obtained by SystemML for mean are consistently better than naive ``sum divided by count" method.

%We also compare the accuracy of mean algorithms in Table~\ref{tab:sum}. Again, the mean algorithm in SystemML consistently produces more accurate results than the naive approach (naive recursive sum divided by count).

\begin{table*}[thb]
\caption{Numerical Accuracy of Higher-Order Statistics (LRE values)}
\label{tab:univariate}
\centering
\begin{tabular}{|c|c|r|r|r|r|r|r|r|r|}
\hline
 & \textbf{Size} & \multicolumn{2}{|c|}{\textbf{Variance}} & \multicolumn{2}{|c|}{\textbf{Std}} & \multicolumn{2}{|c|}{\textbf{Skewness}} & \multicolumn{2}{|c|}{\textbf{Kurtosis}}\\
 & (million) & \multicolumn{2}{|c|}{} & \multicolumn{2}{|c|}{} & \multicolumn{2}{|c|}{} & \multicolumn{2}{|c|}{}\\
\hline
\textbf{Range} & & SystemML & Naive & SystemML & Naive & SystemML & Naive & SystemML & Naive\\ 
\hline
             & 10   & 16.0 & 11.3 & 15.9 & 11.6 & 16.4 & 7.5 & 15.3 & 9.8 \\
\textbf{R1}  & 100  & 16.2 & 11.5 & 16.8 & 11.8 & 14.9 & 7.1 & 15.6 & 9.3 \\
             & 1000 & 16.0 & 11.3 & 16.4 & 11.6 & 14.5 & 6.5 & 15.6 & 8.9 \\
\hline
\hline
             & 10   & 15.4 & 5.9  & 15.9 & 6.2  & 12.5 & 0   & 14.9 & 0 \\
\textbf{R2}  & 100  & 15.6 & 5.3  & 15.8 & 5.6  & 12.0 & 0   & 14.9 & 0 \\
             & 1000 & 16.2 & 4.9  & 16.4 & 5.2  & 12.1 & 0   & 15.2 & 0 \\
\hline
\hline
             & 10   & 14.4 & 0    & 14.7 & 0    & 9.1  & 0   & 12.6  & 0 \\
\textbf{R3}  & 100  & 12.9 & 0    & 13.2 & NA   & 9.0  & NA  & 13.2  & NA \\
             & 1000 & 13.2 & 0    & 13.5 & NA   & 9.4  & NA  & 12.9 & NA \\
\hline
%\multicolumn{10}{l}{}\\
\multicolumn{10}{c}{NA represents undefined standard deviation, skewness or kurtosis due to a negative value for variance.}
\end{tabular}
\end{table*}

The accuracy comparison for higher-order statistics is shown in Table~\ref{tab:univariate}. In SystemML, we employ the algorithms presented in Section~\ref{sec:highorder} to compute required central moments, which are then used to compute higher-order statistics. We compare SystemML against the naive textbook one-pass methods (see Table~\ref{tab:1-pass}). As shown in Table~\ref{tab:univariate}, SystemML attains more accurate results for all statistics. The difference between the two methods in case of higher-order statistics is much more pronounced than that observed for sum and mean from Table~\ref{tab:sum}. This is because the round-off and truncation errors get magnified as the order increases. It is important to note that for some data sets in ranges R2 and R3, the higher-order statistics computed by the naive method are grossly erroneous ($0$ digits matched). More importantly, in some cases, the naive method produced negative values for variance, which led to undefined values for standard deviation, skewness and kurtosis (shown as NA in Table~\ref{tab:univariate}). 

%It is clearly shown in Table~\ref{tab:univariate} that SystemML algorithms are significantly more stable than the textbook one-pass algorithms. The advantage of SystemML is much higher in the higher-order statistics than sum and mean, because errors get more amplified as the order increases. For some datasets in ranges R2 and R3, the higher order statistics are completely wrong (0 digits matched). In particular, the R3 100 million data set and R3 1 billion data set produce negative variance, which leads to undefined standard deviation, skewness and kurtosis (shown as NA in Table~\ref{tab:univariate}).

The value range has a considerable impact on the accuracy of univariate statistics. Even though R1, R2, and R3 have the same delta (i.e., the difference between minimum and maximum value), the accuracies obtained by all the algorithms drop as the magnitude of values increases. A popular technique to address this problem is to shift all the values by a constant, compute the statistics, and add the shifting effect back to the result. The chosen constant is typically the minimum value or the mean (computed or approximate). Chan {\em et al.} showed that such a technique helps in computing statistics with higher accuracy~\cite{variance}. 

%Although the delta (difference between min and max values) of each range is the same (0.5), all algorithms including stable ones present decreasing accuracy from R1, R2 to R3. This is because the 3 ranges have increasing means. If we know ahead of time some basic characteristics of a dataset, we can simply shift the data before computing the statistics, then add the shifting effect back to the results. For example, it has been shown in~\cite{variance} that to compute variance for a dataset with a large mean, substantial gains in accuracy can be achieved by shifting all the data by some approximation of the mean. Similarly, shifting can be applied to the other univariate statistics for increased accuracy.

%\begin{table*}[t]
%\caption{Numerical Accuracy of Univariate Statistics (LRE values)}
%\label{tab:univariate}
%\centering
%\begin{tabular}{|c|c|r|r|r|r|r|r|r|r|r|r|r|r|r|}
%\hline
% & \textbf{size} & \multicolumn{3}{|c|}{\textbf{sum}} &  \multicolumn{2}{|c|}{\textbf{mean}}& \multicolumn{2}{|c|}{\textbf{variance}} & \multicolumn{2}{|c|}{\textbf{std}} & \multicolumn{2}{|c|}{\textbf{skewness}} & \multicolumn{2}{|c|}{\textbf{kurtosis}}\\
% & (million) & \multicolumn{3}{|c|}{} &  \multicolumn{2}{|c|}{}& \multicolumn{2}{|c|}{} & \multicolumn{2}{|c|}{} & \multicolumn{2}{|c|}{} & \multicolumn{2}{|c|}{}\\
%\hline
%\textbf{Range} & & SystemML & naive & sorted & SystemML & naive & SystemML & 1-pass & SystemML & 1-pass & SystemML & 1-pass & SystemML & 1-pass\\ 
%\hline
%    & 10   & 16.1 & 13.5 & 16.1 & 16.7 & 13.5 & 13.5 & 11.3 & 13.8 & 11.6 & 13.0 & 7.5 & 13.7 & 9.8 \\
%\textbf{R1}  & 100  & 16.3 & 13.8 & 13.6 & 15.9 & 13.8 & 13.7 & 11.5 & 14.0 & 11.8 & 12.5 & 7.1 &  12.8 & 9.3 \\
%    & 1000 & 16.8 & 13.6 & 13.5 & 16.5 & 13.6 & 14.1 & 11.3 & 14.4 & 11.6 & 12.1 & 6.5 & 11.8 & 8.9 \\
%\hline
%\hline
%    & 10   & 16.8 & 14.4 & 13.9 & 16.5 & 14.4 & 12.8 & 5.9  & 13.1 & 6.2  & 11.8 & 0   & 13.4 & 0 \\
%\textbf{R2}  & 100  & 16.1 & 13.4 & 13.4 & 15.9 & 13.4 & 12.5 & 5.3  & 12.8 & 5.6  & 9.7  & 0   & 14.3 & 0 \\
%    & 1000 & 16.6 & 13.1 & 13.9 & 16.4 & 13.1 & 13.8 & 4.9  & 14.1 & 5.2  & 9.3  & 0   & 11.8 & 0 \\
%\hline
%\hline
%    & 10   & 15.9 & 14.0 & 13.9 & 15.7 & 14.0 & 9.3  & 0    & 9.6  & 0    & 7.1  & 0   & 9.5  & 0 \\
%\textbf{R3}  & 100  & 16.0 & 13.1 & 13.4 & 16.0 & 13.1 & 9.5  & 0    & 9.8  & NA   & 6.8  & NA  & 9.7  & 0 \\
%    & 1000 & 16.3 & 12.9 & 12.2 & 16.1 & 12.9 & 10.3 & 0    & 10.5 & NA   & 6.3  & NA  & 10.0 & 0 \\
%\hline
%\end{tabular}
%\end{table*}



\subsection{Numerical Stability of Bivariate Statistics}\label{sec:bivariate_exp}

We now discuss the numerical accuracy achieved by SystemML for bivariate statistics. We consider two types of statistics: {\em scale-categorical} and {\em scale-scale}. In the former type, we compute {\em Eta}\footnote{Eta is defined as $(1-\frac{\sum\limits_{r=1}^{R}(n_r-1)\sigma_r^2}{(n-1)\sigma^2})^{\frac{1}{2}}$, where $R$ is the number of categories, $n_r$ is the number of data entries per category, $\sigma_r^2$ is the variance per category, $n$ is the total number of data entries, and $\sigma^2$ is the total variance.} and {\em ANOVA-F}\footnote{ANOVA-F is defined as $\frac{\sum\limits_{r=1}^{R}n_r(\mu_r-\mu)^2}{\sum\limits_{r=1}^{R}(n_r-1)\sigma_r^2}.\frac{n-R}{R-1}$, where $R$ is the number of categories, $n_r$ is the number of data entries per category, $\mu_r$ is the mean per category, $\sigma_r^2$ is the variance per category, $n$ is the total number of data entries, and $\mu$ is the total mean.} measures, whereas in the latter case, we compute {\em covariance} and {\em Pearson correlation (R)}\footnote{Pearson-R is defined as $\frac{\sigma_{xy}}{\sigma_x\sigma_y}$, where $\sigma_{xy}$ is the covariance, $\sigma_x$ and $\sigma_y$ are standard deviations.}. For scale variables, we use data sets in value ranges R1, R2 and R3 that were described in Section~\ref{sec:exp-setup}. For categorical variables, we generated data in which $50$ different categories are uniformly distributed.

For computing these statistics, SystemML relies on numerically stable methods for {\em sum}, {\em mean}, {\em variance} and {\em covariance} from Section~\ref{sec:stability}, whereas the naive method in comparison uses the naive recursive summation for sum, sum divided by count for mean, and textbook one-pass algorithms for variance and covariance. 

The LRE values obtained for scale-categorical statistics are shown in Table~\ref{tab:bivar_sc}. From the table, it is evident that the statistics computed in SystemML have higher accuracy than the ones from the naive method. It can also be observed that the accuracy achieved by both methods reduces as the magnitude of input values increases -- e.g., LRE numbers for R3 are smaller than those of R1. As we move from R1 to R3, the accuracy of the naive method drops more steeply compared to SystemML. This is because the inaccuracies of total and per-category $mean$ and $variance$ quickly propagate and magnify the errors in Eta and ANOVA-F. 
%This emphasizes the fact that the use of inaccurate values in computations can lead to grossly erroneous results. 
Similar trends can be observed in case of covariance and Pearson correlation, as shown in Table~\ref{tab:bivar_ss}. For the cases of R2 vs. R3 with $100$ million and $1$ billion data sets, the naive algorithm produces negative values for variance (see Table~\ref{tab:univariate}), which resulted in undefined values for Pearson-R (shown as NA in Table~\ref{tab:bivar_ss}).

\begin{table}[t]
\centering
\caption{Numerical Accuracy of Bivariate Scale-Categorical Statistics: Eta and ANOVA-F (LRE values)}
\label{tab:bivar_sc}
\begin{tabular}{|c|c|r|r|r|r|}
\hline
 & \textbf{Size} & \multicolumn{2}{|c|}{\textbf{Eta}} &  \multicolumn{2}{|c|}{\textbf{ANOVA-F}} \\
 & (million) & \multicolumn{2}{|c|}{} &  \multicolumn{2}{|c|}{} \\
\hline
\textbf{Range} & & SystemML & Naive & SystemML & Naive \\ 
\hline
             & 10   & 16.2 & 13.7 & 16.2 & 10.0 \\
\textbf{R1}  & 100  & 16.6 & 13.7 & 15.6 & 10.0 \\
             & 1000 & 16.5 & 13.6 & 15.8 & 9.9  \\
\hline
\hline
             & 10   & 16.2 & 7.2 & 13.3 & 3.5 \\
\textbf{R2}  & 100  & 16.6 & 7.4 & 13.4 & 3.7 \\
             & 1000 & 16.5 & 7.9 & 13.4 & 4.3 \\
\hline
\hline
             & 10   & 16.2 & 0   & 10.2 & 0 \\
\textbf{R3}  & 100  & 15.9 & 1.9 & 10.0 & 0 \\
             & 1000 & 16.5 & 1.2 & 10.0 & 0 \\
\hline
\end{tabular}
\end{table}


\begin{table}[t]
\centering
\caption{Numerical Accuracy of Bivariate Scale-Scale Statistics: Covariance and Pearson-R (LRE values)}
\label{tab:bivar_ss}
\begin{tabular}{|c|c|r|r|r|r|}
\hline
 & \textbf{Size} & \multicolumn{2}{|c|}{\textbf{Covariance}} &  \multicolumn{2}{|c|}{\textbf{Pearson-R}} \\
 & (million) & \multicolumn{2}{|c|}{} &  \multicolumn{2}{|c|}{} \\
\hline
\textbf{Range} & & SystemML & Naive & SystemML & Naive \\ 
\hline
                     & 10   & 15.0 &  8.4 & 15.1 &  6.2 \\
\textbf{R1 vs. R2}   & 100  & 15.6 &  8.5 & 15.4 &  6.4 \\
                     & 1000 & 16.0 &  8.7 & 15.7 &  6.2 \\
\hline
\hline
                     & 10   & 13.5 &  3.0 & 13.5  & 3.0  \\
\textbf{R2 vs. R3}   & 100  & 12.8 &  2.8 & 12.7 & NA \\
                     & 1000 & 13.6 &  3.9 & 13.8 & NA \\
\hline
%\multicolumn{6}{l}{}\\
\multicolumn{6}{c}{NA represents undefined Pearson-R due to a negative value for variance.}
\end{tabular}
\end{table}






\subsection{The Impact of \textsc{KahanIncrement}}
\label{sec:comparison_exp}
In this section, we evaluate the effect of using \textsc{KahanIncrement} in the update rules of central moments (Equation~\ref{eq:cmeq}) and covariance (Equation~\ref{eq:coveq}) on the accuracy of the results.

Table~\ref{tab:univariate-vs-hive} and Table~\ref{tab:bivar_vs_hive} show the numerical accuracy achieved by the update rules using \textsc{KahanIncrement} and basic addition for higher-order statistics and scale-scale bivariate statistics. Evidently, update rules using \textsc{KahanIncrement} are able to produce more accurate results for all statistics across the board. In SystemML, the correction terms maintained in Kahan technique helps in reducing the effect of truncation errors. 

%The latest version of HIVE (0.7.1) provides a number descriptive statistics including \textit{sum}, \textit{mean}, \textit{variance}, \textit{standard deviation}, \textit{covariance}, and \textit{pearson correlation}. We now compare the accuracies achieved by SystemML against those of HIVE, for several statistics. Since HIVE (as well as PIG) computes \textit{sum} and \textit{mean} using the naive recursive summation algorithm, the comparison between SystemML and HIVE on these two statistics will be same as the results presented in Table~\ref{tab:sum}. 
%
%For higher order statistics, covariance, and Pearson correlation, both SystemML and HIVE employ the same incremental update rules, as shown in Sections~\ref{sec:highorder}~\&~\ref{sec:covariance}. The primary difference is that SystemML makes use of \textsc{KahanIncrement} to perform aggregations, whereas HIVE relies on the basic addition. The differences in accuracy achieved by SystemML and HIVE are shown in Table~\ref{tab:univariate-vs-hive} and Table~\ref{tab:bivar_vs_hive}. Evidently, SystemML is able to produce more accurate results than HIVE for all statistics across the board. In SystemML, the correction terms maintained in Kahan technique helps in reducing the effect of truncation errors.

%As discussed in Section~\ref{sec:highorder} and Section~\ref{sec:covariance}, HIVE employs similar update rule as in SystemML to compute variance, standard deviation, covariance and pearson correlation. However, SystemML uses \textsc{KahanIncrement} in the update rule, whereas HIVE just employs the basic addition. In Table~\ref{tab:univariate-vs-hive} and Table~\ref{tab:bivar_vs_hive}, we compare the accuracy of results produced by the algorithms used in SystemML and HIVE. For both univariate and bivariate statistics, it is evident that SystemML presents significant advantage over HIVE.

%\begin{table}[thb]
%\caption{Accuracy Comparision between SystemML and HIVE for Variance and Standard Deviation (LRE values)}
%\label{tab:univariate-vs-hive}
%\centering
%\begin{tabular}{|c|c|r|r|r|r|}
%\hline
% & \textbf{Size} & \multicolumn{2}{|c|}{\textbf{Variance}} & \multicolumn{2}{|c|}{\textbf{Std}} \\
% & (million) & \multicolumn{2}{|c|}{} & \multicolumn{2}{|c|}{}\\
%\hline
%\textbf{Range} & & SystemML & HIVE & SystemML & HIVE \\ 
%\hline
%             & 10   & 16.0 & 13.5 & 15.9 & 13.8  \\
%\textbf{R1}  & 100  & 16.2 & 13.7 & 16.8 & 14.0  \\
%             & 1000 & 16.0 & 14.1 & 16.4 & 14.4  \\
%\hline
%\hline
%             & 10   & 15.4 & 12.8  & 15.9 & 13.1   \\
%\textbf{R2}  & 100  & 15.6 & 12.5  & 15.8 & 12.8   \\
%             & 1000 & 16.2 & 13.8  & 16.4 & 14.1   \\
%\hline
%\hline
%             & 10   & 14.4 & 9.3    & 14.7 & 9.6     \\
%\textbf{R3}  & 100  & 12.9 & 9.5    & 13.2 & 9.8    \\
%             & 1000 & 13.2 & 10.3   & 13.5 & 10.6    \\
%\hline
%\end{tabular}
%\end{table}

\begin{table*}[thb]
\caption{The Effect of \textsc{KahanIncrement} on the Accuracy of Higher-Order Statistics (LRE values)}
\label{tab:univariate-vs-hive}
\centering
\begin{tabular}{|c|c|r|r|r|r|r|r|r|r|}
\hline
 & \textbf{Size} & \multicolumn{2}{|c|}{\textbf{Variance}} & \multicolumn{2}{|c|}{\textbf{Std}} & \multicolumn{2}{|c|}{\textbf{Skewness}} & \multicolumn{2}{|c|}{\textbf{Kurtosis}}\\
 & (million) & \multicolumn{2}{|c|}{} & \multicolumn{2}{|c|}{} & \multicolumn{2}{|c|}{} & \multicolumn{2}{|c|}{}\\
\hline
\textbf{Range} & & Kahan & Basic & Kahan & Basic & Kahan & Basic & Kahan & Basic\\ 
\hline
             & 10   & 16.0 & 13.5 & 15.9 & 13.8 & 16.4 & 13.0 & 15.3 & 13.7 \\
\textbf{R1}  & 100  & 16.2 & 13.7 & 16.8 & 14.0 & 14.9 & 12.5 & 15.6 & 12.8 \\
             & 1000 & 16.0 & 14.1 & 16.4 & 14.4 & 14.5 & 12.1 & 15.6 & 11.8\\
\hline
\hline
             & 10   & 15.4 & 12.8  & 15.9 & 13.1  & 12.5 & 11.8  & 14.9 & 13.4 \\
\textbf{R2}  & 100  & 15.6 & 12.5  & 15.8 & 12.8  & 12.0 & 9.7   & 14.9 & 14.3 \\
             & 1000 & 16.2 & 13.8  & 16.4 & 14.1  & 12.1 & 9.3   & 15.2 & 11.8 \\
\hline
\hline
             & 10   & 14.4 & 9.3    & 14.7 & 9.6  & 9.1  & 7.1   & 12.6  & 9.5 \\
\textbf{R3}  & 100  & 12.9 & 9.5    & 13.2 & 9.8  & 9.0  & 6.8  & 13.2  & 9.7 \\
             & 1000 & 13.2 & 10.3   & 13.5 & 10.6 & 9.4  & 6.3  & 12.9 & 10.0 \\
\hline
\end{tabular}
\end{table*}



\begin{table}[t]
\centering
\caption{The Effect of \textsc{KahanIncrement} on the Accuracy of Covariance and Pearson-R (LRE values)}
\label{tab:bivar_vs_hive}
\begin{tabular}{|c|c|r|r|r|r|}
\hline
 & \textbf{Size} & \multicolumn{2}{|c|}{\textbf{Covariance}} &  \multicolumn{2}{|c|}{\textbf{Pearson-R}} \\
 & (million) & \multicolumn{2}{|c|}{} &  \multicolumn{2}{|c|}{} \\
\hline
\textbf{Range} & & Kahan & Basic & Kahan & Basic \\ 
\hline
                     & 10   & 15.0 &  14.2 & 15.1 &  13.0 \\
\textbf{R1 vs. R2}   & 100  & 15.6 &  13.3 & 15.4 &  13.0 \\
                     & 1000 & 16.0 &  14.6 & 15.7 &  14.2 \\
\hline
\hline
                     & 10   & 13.5 &  10.0 & 13.5  & 9.8  \\
\textbf{R2 vs. R3}   & 100  & 12.8 &  10.0 & 12.7 & 10.5 \\
                     & 1000 & 13.6 &  11.4 & 13.8 & 10.5 \\
\hline
\end{tabular}
\end{table}


\subsection{Performance of Order Statistics}

In this section, we evaluate the scalability of the sort-based order statistics algorithm presented in Section~\ref{sec:sort-order-stats}. Script~\ref{scpt:linearrg} shows how order statistics are expressed in DML language -- more details on the DML syntax can be found in~\cite{systemml}. The script computes the median as well as other quantiles as specified by the vector $P$, from the input data $V$. In this experiment, we fix $P=\{0.1, 0.2, 0.3, 0.4, 0.5, 0.6, 0.7, 0.8, 0.9\}$, and vary the size of $V$ (the value range of data in $V$ is R1). Figure~\ref{fig:orderStatsScale} shows the execution time of this script as the input data size increases. For the given DML script, SystemML is able to identify that multiple different order statistics are computed on the same data set, and it accordingly performs a single sort and then computes the required order statistics. Furthermore, all the specified order statistics are selected simultaneously, in parallel.

\
\begin{script}\label{scpt:linearrg}
A Simple Script of Order Statistics\\
\footnotesize
\texttt{
1:\ \ \# input vector (column matrix)\\
2:\ \ V = read("in/V");\\
3:\ \ \# a vector specifying the desired quantiles \\
4:\ \ P = read("in/P");\\
5:\ \ \# compute median\\
6:\ \ median = quantile(V, 0.5);\\
7:\ \ print("median: ", median);\\
8:\ \ \# compute quantiles\\
9:\ \ Q = quantile(V, P);\\
10:\ write(Q, "out/Q");\\}
\end{script}

\onefigure
{figs/order_stats_scale.eps}
{Execution Time for Script 1}
{fig:orderStatsScale}




