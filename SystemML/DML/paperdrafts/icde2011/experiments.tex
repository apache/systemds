The goals of our experimentation are to study scalability under conditions of varying
data and Hadoop cluster sizes, and the effectiveness of optimizations in SystemML. For this
purpose, we chose GNMF for which similar studies have been conducted recently~\cite{msrwww10}, thereby enabling
meaningful comparisons. Since \systemmltext\ is architected to enable a large class of ML algorithms, we also study 2 other popular ML algorithms, namely linear regression and PageRank.

\subsection{Experimental Setup}\label{sec:exp-setup}

The experiments were conducted with Hadoop 0.20~\cite{hadoop} on two different clusters:
\begin{itemize}

\item{40-core cluster}: The cluster uses 5 local machines as worker nodes. Each machine has 8 cores with hyperthreading enabled, 32 GB RAM and 500 GB storage. We set each node to run 15 concurrent
  mappers and 10 concurrent reducers.

\item{100-core EC2 cluster}: The EC2 cluster has 100 worker nodes.  Each node is an EC2
  small instance with 1 compute unit, 1.7 GB memory and 160 GB storage. Each node is set to run 2
  mappers and 1 reducer concurrently.

\end{itemize}

The datasets are synthetic, and for given dimensionality and sparsity, the data generator creates
random matrices with uniformly distributed non-zero cells. A fixed matrix block size
(c.f. Section~\ref{sec:blocking}) of $1000\times 1000$ is used for all the experiments, except for
the matrix blocking experiments in Section~\ref{sec:optimizations}. For the local aggregator used in CPMM, we use an
in-memory buffer pool of size 900 MB on the 40-core cluster and 500 MB on the 100-core EC2 cluster.

\subsection{Scalability}
\label{sec:scale-expt}
We use GNMF shown in Script~\ref{scpt:gnmf} as a running example to demonstrate
scalability on both the 40-core cluster and the 100-core cluster.

The input matrix V is a sparse matrix with $d$ rows and $w$ columns. We fix $w$ to be 100,000 and vary $d$. We
set the sparsity of V to be 0.001, thus each row has 100 non-zero entries on average. The goal of GNMF
algorithm is to compute dense matrices W of size $d\times t$ and H of size $t\times w$, where
$V\approx W H$. $t$ is set to 10 (As described in Section~\ref{sec:intro} in the context of topic modeling, $t$ is the number of topics.). Table~\ref{tab:matrix-stats} lists the characteristics of V, W and H used in our setup.

\some
{plots/gnmf_cmp_all.eps}
{fig:gnmf-cmp-all}
{plots/gnmf_ec2.eps}
{fig:gnmf-ec2}
{plots/ec2_scale.eps}
{fig:ec2-scale}
{Scalability of GNMF: (a) increasing data size on 40-core cluster, (b) increasing data size on 100-core cluster, (c) increasing data size and cluster size}

\noindent {\bf Baseline single machine comparison:} As a baseline for
comparing \systemmltext, we first run GNMF using 64-bit version of R on a single machine with 64 GB
memory. Figure~\ref{fig:gnmf-cmp-all} shows the execution times for one iteration of the algorithm
with increasing sizes of V. For relatively small sizes of V, R runs very efficiently as the data fits in memory. 
However, when the number of rows in V increases to 10 million (1 billion
non-zeros in V), R runs out of memory, while \systemmltext\ continues to scale.

\noindent {\bf Comparison against best known published result:} \cite{msrwww10} introduces a hand-coded MapReduce implementation of GNMF. We use this MapReduce
implementation as a baseline to evaluate the efficiency of the execution plan generated
by \systemmlit\ as well as study the performance overhead of our generic runtime\footnote {Through
personal contact with the authors of~\cite{msrwww10}, we were informed that all the scalability
experiments for the hand-coded GNMF algorithm were conducted on a proprietary SCOPE cluster with
thousands of nodes, and the actual number of nodes scheduled for each execution was not known.}. For a
fair comparison, we re-implemented the algorithm as described in the paper and ran it on the same
40-core cluster as the \systemmltext\ generated plan. The hand-coded algorithm contains 8 full
MapReduce jobs and 2 map-only jobs, while the execution plan generated by \systemmltext\ consists of
10 full MapReduce jobs. For the hand-coded algorithm, the matrices are all prepared in the required
formats: V is in cell representation, W is in a row-wise representation and H is in a column-wise
representation. For the \systemmltext\ plan, the input matrices are all in block representation with block
size $1000\times 1000$. Figure~\ref{fig:gnmf-cmp-all} shows the performance comparison
of \systemmltext\ with the hand-coded implementation. Surprisingly, the performance
of \systemmltext\ is significantly better than the hand-coded implementation. As the number of
non-zeros increases from 10 million to 750 million, execution time on \systemmlit\ increases steadily from 519 seconds
to around 800 seconds, while execution time for the hand-coded plan increases dramatically from 477 seconds to 4048
seconds! There are two main reasons for this difference. First, \systemmltext\ uses the block
representation for V, W, and H, while in the hand-coded implementation, the largest matrix V is in
cell representation. As discussed in Section~\ref{sec:blocking} and to be demonstrated in
Section~\ref{sec:optimizations}, the block representation provides significant performance advantages over
the cell representation. Second, the hand-coded implementation employs an approach very similar to
CPMM for the two most expensive matrix multiplications in GNMF: \texttt{t(W)\mmult V}
and \texttt{V\mmult t(H)}, but without the local aggregator (see Section~\ref{sec:localagg}). As will
be shown in Section~\ref{sec:optimizations}, CPMM with local aggregation significantly outperforms
CPMM without local aggregation.

\noindent {\bf Scalability on 100-core EC2 cluster:} To test \systemmltext\ on a large cluster, we ran GNMF on a 100-core EC2 cluster. In the first
experiment, we fixed the number of nodes in the cluster to be 100, and ran GNMF by varying the
number of non-zero values from 100 million to 5 billion. Figure~\ref{fig:gnmf-ec2} demonstrates the
scalability of \systemmltext\ for one iteration of GNMF. In the second experiment (shown in Figure~\ref{fig:ec2-scale}), 
we varied the number of worker nodes from 40 to 100 and scaled the problem size proportionally from 800 million non-zero values to 2 billion non-zeros. The
ideal scale-out behavior would depict a flat line in the chart. However, it is impossible to realize this ideal
scale-out behavior due to many factors such as network overheads. Nevertheless, Figure~\ref{fig:ec2-scale} presents a steady increase in execution time 
with the growth in data and cluster size.

Besides scalability, \dmlr\ improves productivity and reduces development time of ML algorithms significantly. For example, GNMF is implemented in 11 lines of DML script, but requires more than 1500 lines of Java code in the hand-coded implementation. Similar observations have been
made in~\cite{boom10} regarding the power of declarative languages in substantially simplifying
distributed systems programming.


%The characteristics of the synthetic matrices are listed in Table~\ref{tab:matrix-stats}. In the
%context of the GNMF algorithm, the input matrix V is a sparse document-word matrix with m rows (m
%documents) and n columns (n words). We fix the dictionary size n to be 100,000 and vary the number
%of documents m. On average, the each document has 100 words, thus the sparsity of the matrix is
%0.1\%. The number of topic we want to find out for V is k, thus W and H are both dense matrices of
%size $m\times k$ and $k \times n$, respectively. We set $k=10$ in our experiment. In the context of
%the linear regression algorithm, V again is a document-word matrix with sparsity 0.1\% and y is a
%dense vector of 1s and -1s indicating whether a document belongs to a certain topic or not. Except
%for the experiments in Section~\ref{sec:block-exp}, all the other experiments use block size
%$1000\times 1000$. Note the square blocking works even in the case when one dimension is less than
%1000 (refer to Section~\ref{sec:mrruntime}).


\subsection{Optimizations}
\label{sec:optimizations}

\begin{table*}[t]
\centering
\caption{Characteristics of Matrices.}
\label{tab:matrix-stats}
\begin{tabular}{|c|c|c|c|c|c|c|c|c|}
\hline
Matrix & X,Y,W & H' & V & W' & S & H\\
\hline
Dimension & $d\times 10$ &  $100,000\times 10$ & $d\times 100,000$ & $10\times d$ & $10\times 10$ & $10\times 100,000$\\
\hline
%Block Size & $1000\times 100$ & $1000\times 1000$ & $1000\times 100$ & $1000\times 1000$ \\
%\hline
Sparsity & 1 & 1& 0.001 & 1 & 1 & 1\\
\hline
\#non zeros & $10d$ & 1 million & $100d$ & $10d$ & 100 & 1 million\\
%\hline
%Size in cell representation & & & &\\
%\hline 
%Size in block representation & & & &\\
\hline
\end{tabular}
\SmallCrunch
\end{table*}

\begin{table*}[t]
\centering
\caption{File Sizes of Matrices for different $d$ (block size is 1000x1000)}
\label{tab:file-size}
\begin{tabular}{|c|c|c|c|c|c|c|c|c|c|c|c|}
\hline
& d (million) & 1 & 2.5 & 5 & 7.5 & 10 & 15 & 20 & 30 & 40 & 50\\
\hline
\hline
V & \# non zero (million) & 100 & 250 & 500 & 750 & 1000 & 1500 & 2000 & 3000 & 4000 & 5000\\
\cline{2-12}
& Size (GB)  & 1.5 & 3.7 & 7.5 & 11.2 & 14.9& 22.4& 29.9 & 44.9 &59.8 & 74.8\\
\hline
\hline
X,Y,W,W' & \# non zero (million) & 10 & 25& 50& 75& 100& 150& 200 & 300 &400 & 500\\
\cline{2-12}
& Size (GB) & 0.077 & 0.191 & 0.382& 0.573 & 0.764 & 1.1 & 1.5 & 2.2 & 3.0 & 3.7\\
\hline
\end{tabular}
\SmallCrunch
\end{table*}

%One optimization in the HOP layer of \systemmlit \ is to choose the low level execution plan based
%on cost model. In this section, we use matrix multiplication as an example to illustrate how to
%choose between CPMM and RMM for a particular matrix multiplication.

%In Section~\ref{sec:mrruntime}, we have introduced two alternatives of matrix multiplication and
%discussed their cost model. In this section,

\noindent{\bf RMM vs CPMM:} We now analyze the performance differences between alternative execution plans for matrix multiplication, RMM and CPMM. We consider three examples from GNMF (Script~\ref{scpt:gnmf}): \texttt{V\mmult
  t(H)}, \texttt{W\mmult (H\mmult t(H))}, and \texttt{t(W)\mmult W}. To focus on
matrix multiplication, we set \texttt{H'=t(H)}, \texttt{S=H\mmult t(H)}, and \texttt{W'=t(W)}. Then the
three multiplications are defined as: \texttt{V\mmult H'}, \texttt{W\mmult S} and \texttt{W'\mmult
  W}. The inputs of these three multiplications have very distinct characteristics as shown in
Table~\ref{tab:matrix-stats}. With $d$ taking values in millions, V is a very large matrix; H' is a
medium sized matrix; W' and W are very tall and skinny matrices; and S is a tiny matrix. We compare execution times for
the two alternative algorithms for the three matrix multiplications in Figures~\ref{fig:mmult-cmp1},
\ref{fig:mmult-cmp2} and \ref{fig:mmult-cmp3}.

Note that neither of the algorithms always outperforms the other with their relative performance depending
on the data characteristics as described below.

%\begin{itemize}
%\item 
For \texttt{V\mmult H'}, due to the large sizes of both V and H', CPMM is the preferred
approach over RMM, because the shuffling cost in RMM increases dramatically with the number of rows
in V. 

%\item 
For \texttt{W\mmult S}, RMM is preferred over CPMM, as S is small enough
to fit in one block, and RMM essentially partitions W and broadcasts S to perform the matrix
multiplication. 

%\item 
For \texttt{W'\mmult W}, the cost for RMM is $\textit{shuffle}(|W'|+|W|)+IO_{dfs}(|W'|+|W|+|S|)$ with a degree of parallelism of only 1, while the
cost of CPMM is roughly $\textit{shuffle}(|W'|+|W|+r|S|)+IO_{dfs}(2r|S|+|W'|+|W|+|S|)$. For CPMM, the degree of parallelization is $d/1000$, which ranges from 1000 to 50000 as $d$
increases from 1 million to 50 million. When $d$ is relatively small, even though the degree of
parallelization is only 1, the advantage of the low shuffle cost makes RMM perform better than
CPMM. However, as $d$ increases, CPMM's higher degree of parallelism makes it outperform
RMM. Overall, CPMM performs very stably with increasing sizes of W' and W.
%\end{itemize}

%The result of \texttt{W'\mmult W} will be a $10\times 10$ dense matrix with the same characteristics as S. 


\some
{plots/mmult_cmp1.eps}
{fig:mmult-cmp1}
{plots/mmult_cmp2.eps}
{fig:mmult-cmp2}
{plots/mmult_cmp3.eps}
{fig:mmult-cmp3}
{Comparing two alternatives of matrix multiplication: (a) \texttt{V\mmult H'}, (b) \texttt{W\mmult S}, (c) \texttt{W'\mmult W}}


\noindent{\bf Piggybacking:} To analyze the impact of piggybacking several lops into a single
MapReduce job, we compare piggybacking to a naive approach, where each lop is evaluated in a
separate MapReduce job. Depending on whether a single lop dominates the cost of evaluating a LOP-Dag,
the piggybacking optimization may or may not be significant. To demonstrate this, we first consider the expression
\texttt{W*(Y/X)} with \texttt{X=W\mmult H\mmult t(H)} and \texttt{Y=V\mmult t(H)}. The matrix
characteristics for X, Y, and W are listed in Tables~\ref{tab:matrix-stats}
and~\ref{tab:file-size}. Piggybacking reduces the number of MapReduce jobs from 2 to 1 resulting in
a factor of 2 speed-up as shown in Figure~\ref{fig:piggyback-binary}. On the other hand, consider
the expression \texttt{W*(V\mmult t(H)/X)} from the GNMF algorithm (step 8), where \texttt{X=W\mmult
  H\mmult t(H)}. While piggybacking reduces the number of MapReduce jobs from 5 to 2, the associated
performance gains are small as shown in Figure~\ref{fig:piggyback-mmult}.

%There are 4 language level matrix operations in this expression. These 4 operations can be packed
%into 2 Mapreduce jobs using piggybacking, compared to 5 MapReduce jobs with the naive approach
%(matrix multiplication requires 2 MapReduce jobs, the other operations require one MapReduce job
%each). shows the execution time for the two approaches. The matrix characteristics for X, H, V, and
%W are listed in Table~\ref{tab:matrix-stats}. We fix the size of matrix H (7.8 MB) and increase the
%size of X, V, and W by changing $d$ from 1 million to 20 million. The sizes of the matrices X, V and
%W for different $d$ values can be found in Table~\ref{tab:file-size}. The experimental results show
%that the benefit of piggybacking for this setting is insignificant, because the expensive matrix
%multiplication between V and t(H) dominates the execution time.

\twosubfigures
{plots/piggyback_binary.eps}
{\texttt{W*(Y/X)}}
{fig:piggyback-binary}
{plots/piggyback_mmult.eps}
{\texttt{W*(V\mmult t(H)/X)}}
{fig:piggyback-mmult}
{Piggybacking or not:}

\noindent{\bf Matrix Blocking:} Table~\ref{tab:cmp-block} shows the effect of matrix blocking
on storage and computational efficiency (time) using the expression \texttt{V\mmult H'}. As a
baseline, the table also includes the corresponding numbers for the cell representation. The matrix
characteristics for V with d=1 million rows and H are listed in Table~\ref{tab:matrix-stats}.  The
execution time for the expression improves by orders of magnitude from hours for the cell
representation to minutes for the block representation. 

The impact of block size on storage requirements varies for sparse and dense matrices. For dense
matrix $H'$, blocking significantly reduces the storage requirements compared to the cell
representation. On the other hand, for sparse matrix $V$, small block sizes can increase the storage
requirement compared to the cell representation, since only a small fraction of the cells are non-zero
per block and the per block metadata space requirements are relatively high.

Figure~\ref{fig:blocking-cmp} shows the performance comparison for different block sizes with increasing matrix
sizes\footnote{Smaller block sizes were ignored in this experiment since they took hours even for 1
  million rows in $V$.}. This graph shows that the performance benefit of using a larger block size
increases as the size of V increases.

%Compared to cell representation, blocking reduces the storage for the sparse matrix V almost 37\%
%for $100\times 100$ block size, and 50\% for $1000\times 1000$ block size. If the block size is too
%small ($10\times 10$ block size), the storage for $V$ increases as the block representation incurs
%per-block bookkeeping, and given that for 0.001 sparsity each $10\times 10$ block has only 1 cell
%value in it, the number of blocks will be roughly the same as the number of non-zero cells. The
%storage savings for dense matrix H' are at approx. 75\% compared to cell representation as there are
%less blocks then cells to store.

%However, the result doesn't imply that we should arbitrarily
%increase the block size. A block should always fit in memory and make sure that there is a
%sufficient number of blocks for parallelism, fully utilizing a given cluster.

%Matrix V has 1 million rows.

\begin{table}[t]
\centering
\caption{Comparison of different block sizes}
\label{tab:cmp-block}
\begin{tabular}{|c|c|c|c|c|}
\hline
Block Size & 1000x1000 & 100x100 & 10x10 & cell\\
\hline
Execution time &117sec & 136sec & 3hr & $>$5hr\\
\hline
Size of V (GB) & 1.5 & 1.9 & 4.8 & 3.0\\
\hline
Size of H' (MB) & 7.8 & 7.9 & 8.1 & 31.0\\
\hline
\end{tabular}
\Crunch
\end{table}

\noindent{\bf Local Aggregator for CPMM:} To understand the performance advantages of using a 
local aggregator (Section~\ref{sec:localagg}), consider the evaluation of \texttt{V\mmult H'} (V is
$d\times w$ matrix, and H' is $w \times t$ matrix). The matrix characteristics for V and H' can be
found in Tables~\ref{tab:matrix-stats} and~\ref{tab:file-size}. We first set $w=100,000$ and
$t=10$. In this configuration, each reducer performs 2 cross products on average, and the ideal
performance gain through local aggregation is a factor of 2. Figure~\ref{fig:mmcj-n} shows the
benefit of using the local aggregator. As $d$ increases from 1 million to 20 million, the speedup
ranges from 1.2 to 2.

We next study the effect of $w$, by fixing $d$ at 1 million and varying $w$ from 100,000 to 300,000. The number of cross products performed in each reducer increases as $w$
increases. Consequently, as shown in Figure~\ref{fig:mmcj-k-size}, the intermediate result of
\mmcjlop\ increases linearly with $w$ when a local aggregator is not deployed. On the other hand,
when a local aggregator is applied, the size of the intermediate result stays constant as shown in
the figure. Therefore, the running time with a local aggregator increases very slowly while without an
aggregator the running time increases more rapidly (see Figure~\ref{fig:mmcj-k}).


\twosubfigures
{plots/blocking_cmp.eps}
{Execution time with different block sizes}
{fig:blocking-cmp}
{plots/mmult_mmcj_n.eps}
{Advantage of local aggregator with increasing d}
{fig:mmcj-n}
{}

\twosubfigures
{plots/mmult_mmcj_k_size.eps}
{intermediate result size}
{fig:mmcj-k-size}
{plots/mmult_mmcj_k.eps}
{execution time}
{fig:mmcj-k}
{CPMM with increasing w:}

%{plots/mmult_mmcj_n_size.eps}
%{The intermediate result sizes with increasing m}
%{fig:mmcj-n-size}

%\subsubsection{Other Optimization strategies}
%
%\textbf{Algebraic Rewrites}
%\textbf{Description}: e.g. \texttt{A\mmult B\mmult C = (A\mmult B)\mmult C = A\mmult (B\mmult C)}. Based on the dimensionality and the sparsity of A, B and C, the optimizer will choose the right order to perform the operations. 
%
%\textbf{Conditional Evaluation}


\subsection{Additional Algorithms}
\label{sec:more-algs}
In this section, we showcase another two classic algorithms written in DML: Linear Regression and PageRank~\cite{PageRank}. 

{\bf Linear Regression:} Script~\ref{scpt:linearrg} is an implementation of a conjugate gradient
solver for large, sparse, regularized linear regression problems. In the script below, $V$ is a data
matrix (sparsity 0.001) whose rows correspond to training data points in a high-dimensional, sparse
feature space. The vector $b$ is a dense vector of regression targets. The output vector $w$ has the
learnt parameters of the model that can be used to make predictions on new data points.

\vspace{0.25cm}
\begin{script}\label{scpt:linearrg}
Linear Regression\\
\footnotesize
\texttt{
1:\ V=readMM("in/V", rows=1e8, cols=1e5, nnzs=1e10);\\
2:\ y=readMM("in/y", rows=1e8, cols=1);\\
3:\ lambda  = 1e-6; // regularization parameter \\
4:\ r=-(t(V) \%*\% y) ;\\
5:\ p=-r ;\\
6:\ norm\_r2=sum(r*r);\\
7:\ max\_iteration=20;\\
8:\ i=0;\\
9:\ \textbf{while}(i<max\_iteration)\{\\
10:\ \ q=((t(V) \%*\% (V \%*\% p)) + lambda*p) \\
11:\ \ alpha= norm\_r2/(t(p)\mmult q);\\
12:\ \ w=w+alpha*p;\\
13:\ \ old\_norm\_r2=norm\_r2;\\
14:\ \ r=r+alpha*q;\\
15:\ \ norm\_r2=sum(r*r);\\
16:\ \ beta=norm\_r2/old\_norm\_r2; \\
17:\ \ p=-r+beta*p;\\
18:\ \ i=i+1;\}\\
19:writeMM(w, "out/w");}
\end{script}
\vspace{0.25cm}

{\bf PageRank:} Script~\ref{scpt:pagerank} shows the DML script for the PageRank algorithm. In
this algorithm, $G$ is a row-normalized adjacency matrix (sparsity 0.001) of a directed graph.  The
procedure uses power iterations to compute the PageRank of every node in the graph.

Figures~\ref{fig:linearrg-local} and~\ref{fig:pagerank-local} show the scalability of \systemmltext\ for linear regression and PageRank, respectively. For linear regression, as the number of rows increases from 1 million to 20 million (non-zeros ranging from 100 million to 2 billion), the execution time increases steadily. The PageRank algorithm also scales nicely with increasing number of rows from 100 thousand to 1.5 million (non-zeros ranging from 100 million to 2.25 billion). 

\vspace{0.25cm}
\begin{script}\label{scpt:pagerank}
PageRank\\
\footnotesize
\texttt{
1:\ G=readMM("in/G", rows=1e6, cols=1e6, nnzs=1e9);\\
//p: initial uniform pagerank\\
2:\ p=readMM("in/p", rows=1e6, cols=1); \\
//e: all-ones vector  \\
3:\ e=readMM("in/e", rows=1e6, cols=1); \\
//ut: personalization \\
4:\ ut=readMM("in/ut", rows=1, cols=1e6); \\
5:\ alpha=0.85; //teleport probability \\
6:\ max\_iteration=20;\\
7:\ i=0;\\
8:\ \textbf{while}(i<max\_iteration)\{\\
9:\ \ p=alpha*(G\mmult p)+(1-alpha)*(e\mmult ut\mmult p);\\
10:\ i=i+1;\}\\
11:writeMM(p, "out/p");}
\end{script}
\vspace{0.25cm} 

\twosubfigures
{plots/linearrg_local.eps}
{Execution of Linear Regression with increasing data size on 40-core cluster}
{fig:linearrg-local}
{plots/pagerank_local.eps}
{Execution of PageRank with increasing data size on 40-core cluster}
{fig:pagerank-local}
{}

