
%A popular tool amongst machine learning practitioners is R~\cite{R}.
%The language in R exposes a rich set of
%features including mathematical constructs such as linear algebra operations, 
%control structures such as while and for loops, function support including
%recursive functions and  object-oriented programming features. 
%While it provides the expressivity needed
%to implement a wide variety of ML algorithms, it is procedural in
%nature thereby precluding optimizations that consider 
%alternative plans such as those discussed in Example~\ref{algo:gnmf}.

\dmlr\ is a declarative language whose syntax closely resembles 
the syntax of R\footnote{R is prototypical for a larger class of such languages including
Matlab~\cite{matlab}}~\cite{R}. To enable more system generated optimization, \dmlr\ does not
provide all the flexibility available in R. However, this loss in flexibility results largely in
loss in programming convenience and does not significantly impact the class of ML algorithms that
are expressible in \dmlr.
%We begin by describing the features in \dmlr\ and then proceed to discuss the differences 
%with respect to R~\footnote{R is prototypical of a larger class of such languages 
%including Matlab~\cite{matlab}}. 
The GNMF algorithm (Algorithm~\ref{algo:gnmf}) is expressed in
\dmlr\ syntax in Script~\ref{scpt:gnmf}. We explain \dmlr\
constructs using this example.
\\ 
\begin{script}\label{scpt:gnmf}
GNMF\\
\footnotesize
\texttt{1:\ V=readMM("in/V", rows=1e8, cols=1e5, nnzs=1e10);\\
2:\ W=readMM("in/W", rows=1e8, cols=10);\\
3:\ H=readMM("in/H", rows=10, cols=1e5);\\
4:\ max\_iteration=20;\\
5:\ i=0;\\
6:\ \textbf{while}(i<max\_iteration)\{\\
7:\ \ \ H=H*(t(W)\mmult V)/(t(W)\mmult W\mmult H);\\
8:\ \ \ W=W*(V\mmult t(H))/(W\mmult H\mmult t(H));\\
9:\ \ \ i=i+1;\}\\
10:writeMM(W, "out/W");\\
11:writeMM(H, "out/H");}
\end{script}
\vspace*{0.1in}

\noindent {\bf Data Types:} \dmlr\ supports two main data types: matrices and 
scalars~\footnote{We treat vectors as a special case of matrices.}. Scalar 
data types supported are integer, double, string and logical. The cells in a matrix may 
consist of integer, double, string or logical values.

\noindent {\bf Statements:} A \dmlr\ program consists of a sequence of statements, with 
the default computation semantics being sequential evaluation of the individual 
statements. 

The following constructs are currently supported in \dmlr.

{\it Input/Output}: \concept{ReadMM} and \concept{WriteMM}
statements are provided for respectively reading and writing matrices from and to files. 
Optionally, in the \concept{ReadMM} statement,
the user can provide additional properties of the matrix
such as sparsity (number of non-zero entries or nnzs).

{\it Control Structures}: Control structures supported in \dmlr\
include the \concept{while} statement, \concept{for} statement and
\concept{if} statement. Steps~6-9 in Script~\ref{scpt:gnmf} show an
example \concept{while} statement.

{\it Assignment}: An \concept{assignment} statement 
consists of an expression and the result of which is assigned to a
variable - e.g., Steps~7 ,8 and 9 in Script~\ref{scpt:gnmf}. Note
that the assignment can be to a scalar or a matrix. 

%\begin{itemize}
%\item {\it Input/Output}: \concept{ReadMM} and \concept{WriteMM}
%statements are provided for respectively reading and writing, vectors
%or matrices, from and to files. Optionally, in the \concept{ReadMM} statement,
%the user can provide additional properties of the matrix or vector,
%such as dimensionality and sparsity (number of non-zero entries or nnzs).
%
%\item {\it Control Structures} : Control structures supported in \dmlr\
%include the \concept{while} statement, \concept{for} statement and
%\concept{if} statement. Steps~6-9 in Script~\ref{scpt:gnmf} show an
%example \concept{while} statement.
%
%\item {\it Assignment}: An \concept{assignment} statement 
%consists of an expression and the result of which is assigned to a
%variable - e.g., Steps~7,8 and 9 in Script~\ref{scpt:gnmf}. Note
%that the assignment can be to a scalar, vector or matrix. 
%\end{itemize}

%Table~\ref{tab:operators} lists the main types of operators supported
%in \dmlr\ along with a few examples for each type. The arithmetic
%operators extend naturally to matrices and vectors where the semantics
%is such that the operator is applied to the corresponding cells. For
%instance the expression C = \binaryop{A}{*}{B} will multiply the
%values in the corresponding cells in A and B and populate the
%appropriate cell in C with the result. Several internal functions,
%specific to particular data types, are
%supported. E.g., \concept{colSum} computes the sum of every column in
%a matrix and returns a vector, while \concept{t} computes the
%transpose of a matrix.

Table~\ref{tab:notation} lists several example operators allowed in expressions in \dmlr. 
The arithmetic operators $+,-,*,/$ extend naturally to matrices where the semantics is 
such that the operator is applied to the corresponding cells. For instance, the expression 
$Z=X*Y$ will multiply the values in the corresponding cells in $X$ and $Y$, and populate 
the appropriate cell in $Z$ with the result. Several internal functions, specific to 
particular data types, are supported -- e.g., \concept{rowSum} computes the sum of every 
row in a matrix and returns a column matrix (i.e., a vector), while \concept{t($\cdot$)} 
computes the transpose of a matrix.

\dmlr\ also allows users to define their own functions using the syntax {\textit 
{``function~~(arglist)~~body''}}. Here, the \textit{arglist} consists of a set of formal 
input and output arguments and the body is a group of valid \dmlr\ statements.

\noindent {\bf Comparison with R programming language:} As pointed out before, we have 
made some choices in the design of \dmlr\ to better enable system optimizations. For 
example, \dmlr\ does not support \emph{object oriented features}, \emph{advanced data 
types} (such as lists and arrays) and \emph{advanced function support} (such as accessing 
variables in the caller function and further up in the call-stack). Besides these advanced 
features for programming convenience, R also supports extensive \emph{graphical 
procedures} that are clearly beyond the scope of \dmlr.

%\emph{[[Note: I also suggest that at the end of the paper
%in the discussion section we have a paragraph on ``Revisiting Design Choices of \dmlr\ ``.]]}


\noindent {\bf Program Analysis:}
We now describe the sequence of steps a \dmlr\ script goes through to generate a parsed 
representation. Figure~\ref{fig:programanalysis} shows the result of program analysis for 
Script~\ref{scpt:gnmf}.

{\it Type Assignment}: The first step is to assign data types to
each variable in the \dmlr\ script. For instance, \concept{ReadMM}
statements (Steps~1-3) are used to type V, W and H as matrices, while \concept{Assignment} 
statements (Steps~4-5) are used to identify $max\_iteration$ and $i$ as scalar variables. 
%This information is then propagated through the rest of the program.

{\it Statement Block Identification}: As control constructs (such as \concept{while}) and 
\concept{functions} break the sequential flow of a \dmlr\ program, they naturally divide 
the program into \textit{statement blocks}. Each statement block consists of consecutive 
\concept{Assignment}, \concept{ReadMM} and \concept{WriteMM} statements, as the operations 
involved in these statements can be collectively optimized. 
Figure~\ref{fig:programanalysis} illustrates our example algorithm broken down into three 
statement blocks ($SB_1, SB_2$ and $SB_3$).

%\item {\it Live Variable Analysis}. Notice how variables may be
%updated-in-place in a \dmlr\ program. E.g., in Step~8, the variable W
%is both read and written in the same statement. Since this statement
%appears within a \concept{while} loop, W may either refer to the
%output of Step~2 (for the first iteration) or Step~8 in the previous
%iteration (for subsequent iterations). Furthermore, we need to decide
%what variables need to be passed across statement blocks. Therefore,
%we perform live-variable analysis~\footnote{This is similar to the
%classic data flow analysis typically performed by compilers in
%programming languages} on the program to connect each variable use
%with the previous write for that variable. For each statement block,
%we also identify the variables that will be required from previous
%statement blocks ({\it Live Variables In}) and the variables that will
%be output by the current statement block ({\it Live Variables
%Out}). The results of live variable analysis are shown in
%Figure~\ref{fig:programanalysis}.

{\it Live Variable Analysis}: The goal of this step is twofold:
(a)~Connect each variable use with the immediately preceding write(s)
for that variable across different evaluation paths. For example,
variable W used in Step~7 refers to the output of Step~2 for the
first iteration of the loop and Step~8 for second iteration onwards. (b)~For each statement block, identify the variables that will be required from previous
statement blocks ({\it Live Variables In}) and the variables that will
be output by the current statement block ({\it Live Variables Out}). 
The results of live variable analysis are shown in Figure~\ref{fig:programanalysis}.




%{\tt Sekar, you need to explain: program, statement block, variable analysis, damn across statement blocks.}
