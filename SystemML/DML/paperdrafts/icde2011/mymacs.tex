% ----------------------------------------------------------------
%changes to the algorithm format
%\newcommand{\algorithmicreturn}{\textbf{return}}
%\newcommand{\algorithmiccomment}[1]{// #1}
\newcommand{\DoubleSpace}{\edef\baselinestretch{1.45}\Huge}
% \newcommand{\DoubleSpace}{\edef\baselinestretch{1.0}\Huge\normalsize}
\newcommand{\SingleSpace}{\edef\baselinestretch{1.0}\Huge\normalsize}
\newcommand{\eat}[1] {}
\newcommand{\reminder}[1]{\textcolor{red}{\em $\rightarrow$ (#1)$\leftarrow$}}
\newcommand{\remove}[1]{\textcolor{blue}{\em $\rightarrow$ [{\bf REM-START} (#1) {\bf REM-END}] $\leftarrow$}}

% \newcommand{\reminder}[1]{ [[[ \marginpar{\mbox{$<==$}} #1 ]]] }
% \newcommand{\reminder}[1]{ [[[{\underline{ \sl $\longrightarrow$ #1 $\longleftarrow$}}]]]}
%\newcommand{\BigCrunch}{}
%\newcommand{\Crunch}{}
%\newcommand{\SmallCrunch}{}
%\newcommand{\Skip}{}

\newcommand{\BigCrunch}{\vspace*{-1.5em}}
\newcommand{\Crunch}{\vspace*{-1em}}
\newcommand{\SmallCrunch}{\vspace*{-1ex}}
\newcommand{\Skip}{\vspace*{1ex}}


\newcommand{\myDefnBegin}[1]{
\SmallCrunch
\begin{defn} \label{#1}
}

\newcommand{\myDefnEnd}{
\end{defn}
\SmallCrunch
}

\newcommand{\myThmBegin}[1]{
%\Crunch
\begin{thm} \label{#1}
}

\newcommand{\myThmEnd}{
\end{thm}
%\BigCrunch
}

\newcommand{\customizedfig}[4]{
\begin{figure}[t]
\begin{center}
  \includegraphics[width=#4]{#1}\\
%  \vspace*{-1em}
  \caption{#2}\label{#3}
%  \vspace*{1em}
\end{center}
\BigCrunch
\end{figure}
}

\newcommand{\customizedfigInCol}[4]{
\begin{figure*}[t]
\begin{center}
  \includegraphics[width=#4]{#1}\\
%  \vspace*{-1em}
  \caption{#2}\label{#3}
%  \vspace*{1em}
\end{center}
\BigCrunch
\end{figure*}
}

\newcommand{\onesmallfigure}[3]{
%\Crunch
\begin{figure}[tb]
\begin{center}
\centerline{\epsfxsize=1.2in \epsffile{#1}}
% \DoubleSpace
%\SmallCrunch
\centerline{\parbox{7in}{\caption{#2} \label{#3}}}
%\BigCrunch
\end{center}
\BigCrunch
\end{figure}
}

\newcommand{\onemediumfigure}[3]{
%\Crunch
\begin{figure}[tb]
\begin{center}
\centerline{\epsfxsize=2.0in \epsffile{#1}}
% \DoubleSpace
%\Crunch
\centerline{\parbox{3in}{\caption{#2} \label{#3}}}
% \SingleSpace
\end{center}
\BigCrunch
\end{figure}
}


\newcommand{\onefigure}[3]{
\begin{figure}[t]
\begin{center}
\centerline{\epsfxsize=2.5in \epsffile{#1}}
\centerline{\parbox{7in}{\caption{#2} \label{#3}}}
%\SmallCrunch
\end{center}
\BigCrunch
\end{figure}
}

\newcommand{\onelargefigure}[3]{
% \Crunch
\begin{figure}[t]
\begin{center}
\centerline{\epsfxsize=3.5in \epsffile{#1}}
% \DoubleSpace
\centerline{\parbox{7in}{\caption{ #2} \label{#3}}}
% \SingleSpace
%\SmallCrunch
\end{center}
\BigCrunch
\end{figure}
}

\newcommand{\twofigs}[6]{
\begin{figure*}[t]
\centerline{
            \epsfxsize=3.1in \epsffile{#1}
            \hfill
            \epsfxsize=3.1in \epsffile{#4}
           }
\centerline{
            \parbox{3.1in}{\caption{#2} \label{#3}}
            \hfill
            \parbox{3.1in}{\caption{#5} \label{#6}}
           }
\end{figure*}
}

\newcommand{\twofiguresinonecol}[6]{
\begin{figure}[tb]
\centerline{
            \hfill
            \epsfxsize=2.0in \epsffile{#1}
            \hfill
            \epsfxsize=0.9in \epsffile{#4}
            \hfill
           }
%           \SmallCrunch
\centerline{
            \hfill
            \parbox{2.0in}{\caption{ #2} \label{#3}}
            \hfill
            \parbox{1.1in}{\caption{ #5} \label{#6}}
            \hfill
           }
\SmallCrunch
\end{figure}
% \Skip
}

\newcommand{\twobigfiguresinonecol}[6]{
\begin{figure}
\centerline{
            \hfill
            \epsfxsize=1.8in \epsffile{#1}
            \hfill
            \epsfxsize=1.8in \epsffile{#4}
            \hfill
           }
%           \SmallCrunch
\centerline{
            \hfill
            \parbox{1.72in}{\caption{ #2} \label{#3}}
            \hfill
            \parbox{1.7in}{\caption{ #5} \label{#6}}
            \hfill
           }
\BigCrunch
\end{figure}
% \Skip
}

\newcommand{\twofigures}[6]{
\begin{figure*}[tb]
\centerline{
            \hfill
            \epsfxsize=5in \epsffile{#1}
            \hfill
            \epsfxsize=1.2in \epsffile{#4}
            \hfill
           }
%           \SmallCrunch
\centerline{
            \hfill
            \parbox{5.5in}{\caption{#2} \label{#3}}
            \hfill
            \parbox{2in}{\caption{#5} \label{#6}}
            \hfill
           }
\SmallCrunch
\end{figure*}
% \Skip
}

\def\papernumber #1 raised #2 {
   % \vspace{-#2}
    \vbox to 0pt{\hfill\framebox{\bf Paper Number #1}}
   % \vspace{#2}
}

\newcommand{\threefiguresmss}[9]{
    \begin{figure*}[tbh]
    \centerline{
    \epsfxsize=3in \epsffile{#1}
    \hfill
    \epsfxsize=1.1in \epsffile{#4}
    \hfill
    \epsfxsize=1.3in
    \epsffile{#7}
    }
%\SmallCrunch
%    \SingleSpace
    \centerline{
    \parbox{3.0in}{\caption{#2}
    \label{#3}}
    \hfill
    \parbox{1.8in}{\caption{#5}
    \label{#6}}
    \hfill
    \parbox{1.8in}{\caption{#8}
    \label{#9}}
    }
\BigCrunch
%    \DoubleSpace
\Skip
\end{figure*}
}

\newcommand{\threefiguresmlm}[9]{
    \begin{figure*}[tbh]
    \centerline{
    \epsfxsize=1.90in \epsffile{#1}
    \hfill
    \epsfxsize=2.40in \epsffile{#4}
    \hfill
    \epsfxsize=1.90in
    \epsffile{#7}
    }
%\SmallCrunch
%    \SingleSpace
    \centerline{
    \parbox{2.00in}{\caption{#2}
    \label{#3}}
    \hfill
    \parbox{2.50in}{\caption{#5}
    \label{#6}}
    \hfill
    \parbox{2.00in}{\caption{#8}
    \label{#9}}
    }
\BigCrunch
%    \DoubleSpace
\Skip
\end{figure*}
}

\newcommand{\threefiguressll}[9]{
    \begin{figure*}[tbh]
    \centerline{
    \epsfxsize=1.20in \epsffile{#1}
    \hfill
    \epsfxsize=2.60in \epsffile{#4}
    \hfill
    \epsfxsize=2.5in
    \epsffile{#7}
    }
%\SmallCrunch
%    \SingleSpace
    \centerline{
    \parbox{1.50in}{\caption{#2}
    \label{#3}}
    \hfill
    \parbox{2.60in}{\caption{#5}
    \label{#6}}
    \hfill
    \parbox{2.50in}{\caption{#8}
    \label{#9}}
    }
\BigCrunch
%    \DoubleSpace
%\Skip
\end{figure*}
}

\newcommand{\threefigureslmm}[9]{
    \begin{figure*}[tbh]
    \centerline{
    \epsfxsize=2.8in \epsffile{#1}
    \hfill
    \epsfxsize=2.2in \epsffile{#4}
    \hfill
    \epsfxsize=2.1in
    \epsffile{#7}
    }
%\SmallCrunch
%    \SingleSpace
    \centerline{
    \parbox{2.7in}{\caption{#2}
    \label{#3}}
    \hfill
    \parbox{1.9in}{\caption{#5}
    \label{#6}}
    \hfill
    \parbox{1.9in}{\caption{#8}
    \label{#9}}
    }
\BigCrunch
%    \DoubleSpace
%\Skip
\end{figure*}
}

\newcommand{\threesubfigures}[9]
{
\begin{figure*}[htb]
\centerline{
\subfigure[]{
   \includegraphics[width=1.7in]{#1}
   \label{#3}
\Crunch
 }
 \subfigure[]{
   \includegraphics[width=1.7in]{#4}
   \label{#6}
\Crunch
 }
 \subfigure[]{
   \includegraphics[width=1.7in]{#7}
   \label{#9}
\Crunch
 }
}
\SmallCrunch
\caption{(a) #2, (b) #5, (c) #8}
\Crunch
\end{figure*}
}

\newcommand{\twosubfigures}[7]
{
\begin{figure}[htb]
\centerline{
\subfigure[]{
   \includegraphics[width=1.7in]{#1}
   \label{#3}
\Crunch
 }
 \subfigure[]{
   \includegraphics[width=1.7in]{#4}
   \label{#6}
\Crunch
 }
}
\SmallCrunch
\caption{#7 (a) #2, (b) #5}
\Crunch
\end{figure}
}

\newcommand{\threefigures}[9]{
    \begin{figure*}[tbh]
    \centerline{
    \epsfxsize=1.8in \epsffile{#1}
    \hfill
    \epsfxsize=1.8in \epsffile{#4}
    \hfill
    \epsfxsize=1.8in
    \epsffile{#7}
    }
%\SmallCrunch
%    \SingleSpace
    \centerline{
    \parbox{2.4in}{\caption{#2}
    \label{#3}}
    \hfill
    \parbox{2.4in}{\caption{#5}
    \label{#6}}
    \hfill
    \parbox{2.4in}{\caption{#8}
    \label{#9}}
    }
%\BigCrunch
\Crunch
%    \DoubleSpace
%\Skip
\end{figure*}
}

\newcommand{\some}[7]
{
\begin{figure*}[htb]
\centerline{
\subfigure[]{
   \includegraphics[width=1.7in]{#1}
   \label{#2}
\Crunch
 }
 \subfigure[]{
   \includegraphics[width=1.7in]{#3}
   \label{#4}
\Crunch
 }
 \subfigure[]{
   \includegraphics[width=1.7in]{#5}
   \label{#6}
\Crunch
 }
}
\SmallCrunch
\caption{#7}
\Crunch
\end{figure*}
}

\newcommand{\sllsubfigures}[9]
{
\begin{figure*}[htb]
\centerline{
\subfigure[]{
   \includegraphics[width=1.2in]{#1}
   \label{#3}
\Crunch
 }
 \subfigure[]{
   \includegraphics[width=2.6in]{#4}
   \label{#6}
\Crunch
 }
 \subfigure[]{
   \includegraphics[width=2.5in]{#7}
   \label{#9}
\Crunch
 }
}
\SmallCrunch
\caption{(a) #2, (b) #5, (c) #8}
\Crunch
\end{figure*}
}
