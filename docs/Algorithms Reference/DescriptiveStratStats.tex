\subsection{Stratified Bivariate Statistics}

\noindent{\bf Description}
\smallskip

The {\tt stratstats.dml} script computes common bivariate statistics, such
as correlation, slope, and their p-value, in parallel for many pairs of input
variables in the presence of a confounding categorical variable.  The values
of this confounding variable group the records into strata (subpopulations),
in which all bivariate pairs are assumed free of confounding.  The script
uses the same data model as in one-way analysis of covariance (ANCOVA), with
strata representing population samples.  It also outputs univariate stratified
and bivariate unstratified statistics.

\begin{table}[t]\hfil
\begin{tabular}{|l|ll|ll|ll||ll|}
\hline
Month of the year & \multicolumn{2}{l|}{October} & \multicolumn{2}{l|}{November} &
    \multicolumn{2}{l||}{December} & \multicolumn{2}{c|}{Oct$\,$--$\,$Dec} \\
Customers, millions    & 0.6 & 1.4 & 1.4 & 0.6 & 3.0 & 1.0 & 5.0 & 3.0 \\
\hline
Promotion (0 or 1)     & 0   & 1   & 0   & 1   & 0   & 1   & 0   & 1   \\
Avg.\ sales per 1000   & 0.4 & 0.5 & 0.9 & 1.0 & 2.5 & 2.6 & 1.8 & 1.3 \\
\hline
\end{tabular}\hfil
\caption{Stratification example: the effect of the promotion on average sales
becomes reversed and amplified (from $+0.1$ to $-0.5$) if we ignore the months.}
\label{table:stratexample}
\end{table}

To see how data stratification mitigates confounding, consider an (artificial)
example in Table~\ref{table:stratexample}.  A highly seasonal retail item
was marketed with and without a promotion over the final 3~months of the year.
In each month the sale was more likely with the promotion than without it.
But during the peak holiday season, when shoppers came in greater numbers and
bought the item more often, the promotion was less frequently used.  As a result,
if the 4-th quarter data is pooled together, the promotion's effect becomes
reversed and magnified.  Stratifying by month restores the positive correlation.

The script computes its statistics in parallel over all possible pairs from two
specified sets of covariates.  The 1-st covariate is a column in input matrix~$X$
and the 2-nd covariate is a column in input matrix~$Y$; matrices $X$ and~$Y$ may
be the same or different.  The columns of interest are given by their index numbers
in special matrices.  The stratum column, specified in its own matrix, is the same
for all covariate pairs.

Both covariates in each pair must be numerical, with the 2-nd covariate normally
distributed given the 1-st covariate (see~Details).  Missing covariate values or
strata are represented by~``NaN''.  Records with NaN's are selectively omitted
wherever their NaN's are material to the output statistic.

\smallskip
\pagebreak[3]

\noindent{\bf Usage}
\smallskip

{\hangindent=\parindent\noindent\it%
{\tt{}-f }path/\/{\tt{}stratstats.dml}
{\tt{} -nvargs}
{\tt{} X=}path/file
{\tt{} Xcid=}path/file
{\tt{} Y=}path/file
{\tt{} Ycid=}path/file
{\tt{} S=}path/file
{\tt{} Scid=}int
{\tt{} O=}path/file
{\tt{} fmt=}format

}


\smallskip
\noindent{\bf Arguments}
\begin{Description}
\item[{\tt X}:]
Location (on HDFS) to read matrix $X$ whose columns we want to use as
the 1-st covariate (i.e.~as the feature variable)
\item[{\tt Xcid}:] (default:\mbox{ }{\tt " "})
Location to read the single-row matrix that lists all index numbers
of the $X$-columns used as the 1-st covariate; the default value means
``use all $X$-columns''
\item[{\tt Y}:] (default:\mbox{ }{\tt " "})
Location to read matrix $Y$ whose columns we want to use as the 2-nd
covariate (i.e.~as the response variable); the default value means
``use $X$ in place of~$Y$''
\item[{\tt Ycid}:] (default:\mbox{ }{\tt " "})
Location to read the single-row matrix that lists all index numbers
of the $Y$-columns used as the 2-nd covariate; the default value means
``use all $Y$-columns''
\item[{\tt S}:] (default:\mbox{ }{\tt " "})
Location to read matrix $S$ that has the stratum column.
Note: the stratum column must contain small positive integers; all fractional
values are rounded; stratum IDs of value ${\leq}\,0$ or NaN are treated as
missing.  The default value for {\tt S} means ``use $X$ in place of~$S$''
\item[{\tt Scid}:] (default:\mbox{ }{\tt 1})
The index number of the stratum column in~$S$
\item[{\tt O}:]
Location to store the output matrix defined in Table~\ref{table:stratoutput}
\item[{\tt fmt}:] (default:\mbox{ }{\tt "text"})
Matrix file output format, such as {\tt text}, {\tt mm}, or {\tt csv};
see read/write functions in SystemML Language Reference for details.
\end{Description}


\begin{table}[t]\small\hfil
\begin{tabular}{|rcl|rcl|}
\hline
& Col.\# & Meaning & & Col.\# & Meaning \\
\hline
\multirow{9}{*}{\begin{sideways}1-st covariate\end{sideways}}\hspace{-1em}
& 01     & $X$-column number                & 
\multirow{9}{*}{\begin{sideways}2-nd covariate\end{sideways}}\hspace{-1em}
& 11     & $Y$-column number                \\
& 02     & presence count for $x$           & 
& 12     & presence count for $y$           \\
& 03     & global mean $(x)$                & 
& 13     & global mean $(y)$                \\
& 04     & global std.\ dev. $(x)$          & 
& 14     & global std.\ dev. $(y)$          \\
& 05     & stratified std.\ dev. $(x)$      & 
& 15     & stratified std.\ dev. $(y)$      \\
& 06     & $R^2$ for $x \sim {}$strata      & 
& 16     & $R^2$ for $y \sim {}$strata      \\
& 07     & adjusted $R^2$ for $x \sim {}$strata      & 
& 17     & adjusted $R^2$ for $y \sim {}$strata      \\
& 08     & p-value, $x \sim {}$strata       & 
& 18     & p-value, $y \sim {}$strata       \\
& 09--10 & reserved                         & 
& 19--20 & reserved                         \\
\hline
\multirow{9}{*}{\begin{sideways}$y\sim x$, NO strata\end{sideways}}\hspace{-1.15em}
& 21     & presence count $(x, y)$          &
\multirow{10}{*}{\begin{sideways}$y\sim x$ AND strata$\!\!\!\!$\end{sideways}}\hspace{-1.15em}
& 31     & presence count $(x, y, s)$       \\
& 22     & regression slope                 &
& 32     & regression slope                 \\
& 23     & regres.\ slope std.\ dev.        &
& 33     & regres.\ slope std.\ dev.        \\
& 24     & correlation${} = \pm\sqrt{R^2}$  &
& 34     & correlation${} = \pm\sqrt{R^2}$  \\
& 25     & residual std.\ dev.              &
& 35     & residual std.\ dev.              \\
& 26     & $R^2$ in $y$ due to $x$          &
& 36     & $R^2$ in $y$ due to $x$          \\
& 27     & adjusted $R^2$ in $y$ due to $x$ &
& 37     & adjusted $R^2$ in $y$ due to $x$ \\
& 28     & p-value for ``slope = 0''        &
& 38     & p-value for ``slope = 0''        \\
& 29     & reserved                         &
& 39     & \# strata with ${\geq}\,2$ count \\
& 30     & reserved                         &
& 40     & reserved                         \\
\hline
\end{tabular}\hfil
\caption{The {\tt stratstats.dml} output matrix has one row per each distinct
pair of 1-st and 2-nd covariates, and 40 columns with the statistics described
here.}
\label{table:stratoutput}
\end{table}




\noindent{\bf Details}
\smallskip

Suppose we have $n$ records of format $(i, x, y)$, where $i\in\{1,\ldots, k\}$ is
a stratum number and $(x, y)$ are two numerical covariates.  We want to analyze
conditional linear relationship between $y$ and $x$ conditioned by~$i$.
Note that $x$, but not~$y$, may represent a categorical variable if we assign a
numerical value to each category, for example 0 and 1 for two categories.

We assume a linear regression model for~$y$:
\begin{equation}
y_{i,j} \,=\, \alpha_i + \beta x_{i,j} + \eps_{i,j}\,, \quad\textrm{where}\,\,\,\,
\eps_{i,j} \sim \Normal(0, \sigma^2)
\label{eqn:stratlinmodel}
\end{equation}
Here $i = 1\ldots k$ is a stratum number and $j = 1\ldots n_i$ is a record number
in stratum~$i$; by $n_i$ we denote the number of records available in stratum~$i$.
The noise term~$\eps_{i,j}$ is assumed to have the same variance in all strata.
When $n_i\,{>}\,0$, we can estimate the means of $x_{i, j}$ and $y_{i, j}$ in
stratum~$i$ as
\begin{equation*}
\bar{x}_i \,= \Big(\sum\nolimits_{j=1}^{n_i} \,x_{i, j}\Big) / n_i\,;\quad
\bar{y}_i \,= \Big(\sum\nolimits_{j=1}^{n_i} \,y_{i, j}\Big) / n_i
\end{equation*}
If $\beta$ is known, the best estimate for $\alpha_i$ is $\bar{y}_i - \beta \bar{x}_i$,
which gives the prediction error sum-of-squares of
\begin{equation}
\sum\nolimits_{i=1}^k \sum\nolimits_{j=1}^{n_i} \big(y_{i,j} - \beta x_{i,j} - (\bar{y}_i - \beta \bar{x}_i)\big)^2
\,\,=\,\, \beta^{2\,}V_x \,-\, 2\beta \,V_{x,y} \,+\, V_y
\label{eqn:stratsumsq}
\end{equation}
where $V_x$, $V_y$, and $V_{x, y}$ are, correspondingly, the ``stratified'' sample
estimates of variance $\Var(x)$ and $\Var(y)$ and covariance $\Cov(x,y)$ computed as
\begin{align*}
V_x     \,&=\, \sum\nolimits_{i=1}^k \sum\nolimits_{j=1}^{n_i} \big(x_{i,j} - \bar{x}_i\big)^2; \quad
V_y     \,=\, \sum\nolimits_{i=1}^k \sum\nolimits_{j=1}^{n_i} \big(y_{i,j} - \bar{y}_i\big)^2;\\
V_{x,y} \,&=\, \sum\nolimits_{i=1}^k \sum\nolimits_{j=1}^{n_i} \big(x_{i,j} - \bar{x}_i\big)\big(y_{i,j} - \bar{y}_i\big)
\end{align*}
They are stratified because we compute the sample (co-)variances in each stratum~$i$
separately, then combine by summation.  The stratified estimates for $\Var(X)$ and $\Var(Y)$
tend to be smaller than the non-stratified ones (with the global mean instead of $\bar{x}_i$
and~$\bar{y}_i$) since $\bar{x}_i$ and $\bar{y}_i$ fit closer to $x_{i,j}$ and $y_{i,j}$
than the global means.  The stratified variance estimates the uncertainty in $x_{i,j}$ 
and~$y_{i,j}$ given their stratum~$i$.

Minimizing over~$\beta$ the error sum-of-squares~(\ref{eqn:stratsumsq})
gives us the regression slope estimate \mbox{$\hat{\beta} = V_{x,y} / V_x$},
with~(\ref{eqn:stratsumsq}) becoming the residual sum-of-squares~(RSS):
\begin{equation*}
\mathrm{RSS} \,\,=\, \,
\sum\nolimits_{i=1}^k \sum\nolimits_{j=1}^{n_i} \big(y_{i,j} - 
\hat{\beta} x_{i,j} - (\bar{y}_i - \hat{\beta} \bar{x}_i)\big)^2
\,\,=\,\,  V_y \,\big(1 \,-\, V_{x,y}^2 / (V_x V_y)\big)
\end{equation*}
The quantity $\hat{R}^2 = V_{x,y}^2 / (V_x V_y)$, called \emph{$R$-squared}, estimates the fraction
of stratified variance in~$y_{i,j}$ explained by covariate $x_{i, j}$ in the linear 
regression model~(\ref{eqn:stratlinmodel}).  We define \emph{stratified correlation} as the
square root of~$\hat{R}^2$ taken with the sign of~$V_{x,y}$.  We also use RSS to estimate
the residual standard deviation $\sigma$ in~(\ref{eqn:stratlinmodel}) that models the prediction error
of $y_{i,j}$ given $x_{i,j}$ and the stratum:
\begin{equation*}
\hat{\beta}\, =\, \frac{V_{x,y}}{V_x}; \,\,\,\, \hat{R} \,=\, \frac{V_{x,y}}{\sqrt{V_x V_y}};
\,\,\,\, \hat{R}^2 \,=\, \frac{V_{x,y}^2}{V_x V_y};
\,\,\,\, \hat{\sigma} \,=\, \sqrt{\frac{\mathrm{RSS}}{n - k - 1}}\,\,\,\,
\Big(n = \sum_{i=1}^k n_i\Big)
\end{equation*}

The $t$-test and the $F$-test for the null-hypothesis of ``$\beta = 0$'' are
obtained by considering the effect of $\hat{\beta}$ on the residual sum-of-squares,
measured by the decrease from $V_y$ to~RSS.
The $F$-statistic is the ratio of the ``explained'' sum-of-squares
to the residual sum-of-squares, divided by their corresponding degrees of freedom.
There are $n\,{-}\,k$ degrees of freedom for~$V_y$, parameter $\beta$ reduces that
to $n\,{-}\,k\,{-}\,1$ for~RSS, and their difference $V_y - {}$RSS has just 1 degree
of freedom:
\begin{equation*}
F \,=\, \frac{(V_y - \mathrm{RSS})/1}{\mathrm{RSS}/(n\,{-}\,k\,{-}\,1)}
\,=\, \frac{\hat{R}^2\,(n\,{-}\,k\,{-}\,1)}{1-\hat{R}^2}; \quad
t \,=\, \hat{R}\, \sqrt{\frac{n\,{-}\,k\,{-}\,1}{1-\hat{R}^2}}.
\end{equation*}
The $t$-statistic is simply the square root of the $F$-statistic with the appropriate
choice of sign.  If the null hypothesis and the linear model are both true, the $t$-statistic
has Student $t$-distribution with $n\,{-}\,k\,{-}\,1$ degrees of freedom.  We can
also compute it if we divide $\hat{\beta}$ by its estimated standard deviation:
\begin{equation*}
\stdev(\hat{\beta})_{\mathrm{est}} \,=\, \hat{\sigma}\,/\sqrt{V_x} \quad\Longrightarrow\quad
t \,=\, \hat{R}\sqrt{V_y} \,/\, \hat{\sigma} \,=\, \beta \,/\, \stdev(\hat{\beta})_{\mathrm{est}}
\end{equation*}
The standard deviation estimate for~$\beta$ is included in {\tt stratstats.dml} output.

\smallskip
\noindent{\bf Returns}
\smallskip

The output matrix format is defined in Table~\ref{table:stratoutput}.

\smallskip
\noindent{\bf Examples}
\smallskip

{\hangindent=\parindent\noindent\tt
\hml -f stratstats.dml -nvargs X=/user/biadmin/X.mtx Xcid=/user/biadmin/Xcid.mtx
  Y=/user/biadmin/Y.mtx Ycid=/user/biadmin/Ycid.mtx S=/user/biadmin/S.mtx Scid=2
  O=/user/biadmin/Out.mtx fmt=csv

}
{\hangindent=\parindent\noindent\tt
\hml -f stratstats.dml -nvargs X=/user/biadmin/Data.mtx Xcid=/user/biadmin/Xcid.mtx
  Ycid=/user/biadmin/Ycid.mtx Scid=7 O=/user/biadmin/Out.mtx

}

%\smallskip
%\noindent{\bf See Also}
%\smallskip
%
%For non-stratified bivariate statistics with a wider variety of input data types
%and statistical tests, see \ldots.  For general linear regression, see
%{\tt LinearRegDS.dml} and {\tt LinearRegCG.dml}.  For logistic regression, appropriate
%when the response variable is categorical, see {\tt MultiLogReg.dml}.

